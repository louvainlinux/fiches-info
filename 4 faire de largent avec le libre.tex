%%%%%%%%%%%%%%%%%%%%%%%%%%%%%%%%%%%
% Header pour fiche info du LLNux %
%     Version 1.1 (03/11/2012)    %
%%%%%%%%%%%%%%%%%%%%%%%%%%%%%%%%%%%

\documentclass[a4paper,10pt]{article}

\usepackage[utf8]{inputenc}           % UTF-8
\usepackage[a4paper, dvips]{geometry} % mise en page générale
\usepackage{lmodern}
\usepackage[T1]{fontenc}              % polices
\usepackage{multicol}                 % multi-colonnes avancé
\usepackage{xspace}                   % espaces conditionnels
\usepackage{graphicx}                 % images
\usepackage[francais]{babel}          % français

\usepackage{amsmath}                  % math
\usepackage{amssymb}                  % math
\usepackage{gensymb}                  % math

\usepackage{url}
	\urlstyle{sf}
\usepackage[usenames]{color}
\usepackage{varioref}                 % \vpageref

\usepackage{fancyhdr}                 % cfoot
\pagestyle{fancy}                     % we need info at the bottom but not a footnote
\renewcommand{\headrulewidth}{0pt}    % ça crée une ligne au-dessus... on l'enlève
% \thispagestyle{empty}               % we need only one page

\renewcommand{\familydefault}{\sfdefault}

\definecolor{codeBlue}{rgb}{0,0,1}
\definecolor{webred}{rgb}{0.5,0,0}
\definecolor{codeGreen}{rgb}{0,0.5,0}
\definecolor{codeGrey}{rgb}{0.6,0.6,0.6}
\definecolor{webdarkblue}{rgb}{0,0,0.4}
\definecolor{webgreen}{rgb}{0,0.3,0}
\definecolor{webblue}{rgb}{0,0,0.8}
\definecolor{orange}{rgb}{0.7,0.1,0.1}

% Configuration de babel
\NoAutoSpaceBeforeFDP

% Configuration de graphicx
\graphicspath{{images/}}

% Titre, auteur et date
\title{Informations sur les logiciels libres}
\author{Kot-à-projet Louvain-Li-Nux}
\date{\today}

% Commandes personnalisées
\newcommand{\upcirc}{$^{\circ}$\xspace}
\newcommand{\super}[1]{$^{\mbox{\scriptsize #1}}$\xspace}

% Réduction des marges
% Dimensions de la page :       	

  %%%%%%%%%%%%%%%%%%%%%%%%%%%%%%%%%%%%%%%%  0
  %   |                                  %
  %---+----------------------------------%  1
  %   | +----------------------------+   %  2
  %   | |          en-tête           |   %
  %   | +----------------------------+   %  3
  %   | +----------------------------+   %  4
  %   | |                            |   %       Remarques : 
  %   | |                            |   %        . distance de '0' à '1' : un pouce + \voffset
  %   | |                            |   %        . distance de 'a' à 'b' : un pouce + \hoffset
  %   | |           texte            |   %
  %   | |                            |   %
  %   | |                            |   %
  %   | |                            |   %
  %   | +----------------------------+   %  5
  %   | +----------------------------+   %
  %   | |         bas de page        |   %
  %   | +----------------------------+   %  6
  %%%%%%%%%%%%%%%%%%%%%%%%%%%%%%%%%%%%%%%%
  %a  b c                            d

   % général
     \voffset       -40mm    % pour descendre (si positif) ou remonter (si négatif) le tout
     \hoffset       -12.5mm    % pour agrandir (si positif) ou diminuer (si négatif) la marge gauche
     \oddsidemargin 10pt   % distance de 'b' à 'c'
   % texte
     \headsep       42pt   % distance de '3' à '4', la distance entre l'en-tête et le texte
     \textheight    760pt  % distance de '4' à '5', pour déterminer la hauteur du texte
     \textwidth     177.5mm  % distance de 'c' à 'd' 
   % en-tête
     \topmargin     0pt    % distance de '1' à '2', pour descendre (si positif) ou remonter (si négatif) le tout
     \headheight    14pt   % distance de '2' à '3', doit être > 13.59999
   % bas de page
     \footskip      38pt   % distance de '5' à '6', la distance entre le texte et le bas de page
%\addtolength{\hoffset}{-1cm}
%\addtolength{\textwidth}{2.5cm}
%\addtolength{\voffset}{-2cm}
%\addtolength{\textheight}{4.9cm}

% Logo du Louvain-Li-Nux
\newsavebox{\logollnux}
\sbox{\logollnux}{\raisebox{-2.2cm}{\includegraphics[height=2.5cm]{../logo.png}}}
\newsavebox{\logoqr}
\sbox{\logoqr}{\raisebox{-2.2cm}{\includegraphics[height=2.5cm]{../qr.png}}}

%titre
\newcommand{\titlellnux}[1]
{
	\hspace{-1cm}
	\begin{tabular}{lp{13cm}r}
		\usebox{\logoqr}
			&
		\begin{center}{\Large Les fiches du Louvain-li-Nux\linebreak \linebreak
		\LARGE \textbf{Episode #1}}\end{center}
			&
		\usebox{\logollnux}
	\end{tabular}
	\vspace{-.5cm}
}

% pied de page
\rfoot{\url{www.louvainlinux.be}\hspace{-3.0cm}}
\cfoot{}
\lfoot{Rue Constantin Meunier 12 (Bruyères)}

% PDF
   \usepackage[pdftitle={Fiche Informative},  % apparition ds les propriétés du doc
               pdfauthor={Louvain-li-Nux},
               pdfsubject={Fiche Informative du Louvain-li-Nux},
               pdfkeywords={louvain-li-nux,llnux,fiche info,lln,kap,ucl},
	       colorlinks=true,
	       linkcolor=webdarkblue, 
	       filecolor=webblue, 
	       urlcolor=webdarkblue,
	       citecolor=webgreen]{hyperref}


\begin{document}

% Titre
\begin{tabular}{p{13cm}r}
    \begin{center}{\Large Les fiches du Louvain-li-Nux\linebreak \linebreak
    \LARGE Episode 4: Faire de l'argent avec les logiciels libres}\end{center}
		&
	\usebox{\logollnux}
\end{tabular}

% Contenu

\section*{Comment est-ce possible?}
L'économie servant à produire, distribuer et allouer des ressources rares,
on peut se demander comment des entreprises arrivent à faire de l'argent avec des logiciels libres qui, par définition, sont librement distribuables et donc souvent gratuits.

En général, un logiciel est tout sauf une ressource rare. Il y a quinze ans, les coûts
associés à la distribution d'une copie en plus d'un logiciel (le prix d'un CD et d'une boite) étaient déjà négligeables
comparés aux coûts de développement. Aujourd’hui, avec l'omniprésence de l'internet, ils tendent vers zéro.
Ceci est aussi vrai pour les logiciels propriétaires. La solution adoptée par
la majorité de l'industrie était alors de les raréfier artificiellement.
On a mis en place des systèmes de protection contre le copiage qui sont devenus 
de plus en plus envahissants. Les clients doivent signer des licences onéreuses qui
leur donnent seulement le droit d'utiliser le logiciel dans des circonstances délimitées par le vendeur.
Il est par exemple souvent illégal de louer un logiciel, pourtant légalement acquis,
pour que quelqu'un d'autre puisse le tester.
Les logiciels libres veulent éviter tous ces restrictions, ils encouragent même le copiage.

Mais comment peut-on donc baser un \textit{business model} sur une ressource non rare?                           
La solution à ce dilemme réside dans le terme "ressources" qui
ne se limite pas seulement aux biens mais comporte aussi les services.
Pour comprendre comment ça fonctionne, il faut se concentrer sur le monde des 
des logiciels professionnels utilisés en entreprise. Ici, les coûts de la licence
ne représentent pas la totalité des dépenses. On les achète presque toujours avec
des contrats de maintenance et des formations.
La monétarisation des logiciels libres a donc le désavantage de l'absence des frais de licence
comme partie des revenus mais elle le compense par d'autres avantages que nous allons 
expliquer au moyen de quelques exemples.

\section*{Les contrats de service}
Une entreprise qui veut faire tourner toutes ses bases de données internes (gestion du personnel,
gestion des clients, gestion des stocks, comptabilité, etc.) sur un logiciel a intérêt à bien faire son choix.
Comme il lui sera difficile de changer par après, il s'agit d'une décision stratégique.
Dans tous les cas, elle ne va pas se contenter "d'acheter" ce logiciel et de se débrouiller toute seule.
Typiquement, elle signe un contrat de garantie/service/maintenance. Ce genre de contrat fonctionne indépendamment du
caractère libre ou propriétaire du logiciel en question. Le contrat inclut au-moins l'installation sur place.
En plus, il spécifie une personne de contact et les détails de l'intervention garantie en cas d'urgence.

Tant que l'entreprise cliente peut se tourner vers un "vendeur" fiable qui fait preuve de références 
positives dans le passé, elle ne se souciera pas trop du caractère libre du logiciel.
Pour des exemples concrets, consultez \url{openerp.com} et \url{redhat.com}.

\section*{La programmation sur commande}
La programmation sur la demande d'un client particulier s'est toujours avérée intéressante puisqu'on
peut lui facturer l'entièreté des coûts. Le code ainsi produit reste confidentiel dans le
modèle propriétaire. Dans bien de cas, même l'acheteur ignore le fonctionnement 
interne du programme qu'il a commandé et le vendeur est interdit de l'utiliser dans d'autres contextes.
Ce type de licences emprisonne les résultats du travail et produit des redondances incroyables.

Par contre, quand le logiciel conçu est destiné à être libre, on facture souvent uniquement une partie 
des coûts au client. Ce dernier sera incité à commander son code en open source puisqu'il paye moins.
En même temps, le vendeur peut se permettre de facturer moins cher parce que ce code open source
améliorera son portfolio de solutions disponibles. Il pourra vendre plus de contrats de maintenance.

\section*{Les partenariats}
Les partenariats perfectionnent la synergie entre le producteur et l'utilisateur du logiciel
libre. L'entreprise partenaire s'occupe de tous les services à fournir sur place comme l'installation, la formation et
la traduction de la documentation. Parfois il étend le code source du logiciel pour ajouter des modules. 
Il profite de la crédibilité du producteur et reçoit une partie des revenus du contrat de service pour son travail.
Le producteur se charge de fournir une équipe de base qui programme le noyau du programme et 
prend les décisions stratégiques à long terme. Il est heureux que les partenaires
distribuent son produit. Chacun s'occupe de ce qu'il peut faire le mieux, une situation \textit{win-win}.
Cette collaboration est beaucoup plus difficile
quand il s'agit d'un logiciel propriétaire, la confidentialité freine l'échange et le progrès.
\end{document}
	
	