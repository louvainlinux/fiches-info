\documentclass[12pt,a4paper]{article}

\usepackage[utf8x]{inputenc}          % UTF-8
\usepackage[a4paper, dvips]{geometry} % mise en page générale
\usepackage[T1]{fontenc}              % polices
\usepackage{multicol}                 % multi-colonnes avancé
\usepackage{xspace}                   % espaces conditionnels
\usepackage{graphicx}                 % images
\usepackage[francais]{babel}          % français

\thispagestyle{empty}

% Configuration de babel
\NoAutoSpaceBeforeFDP

% Configuration de graphicx
\graphicspath{{images/}}

% Titre, auteur et date
\title{Informations sur les logiciels libres}
\author{Kot-à-projet Louvain-Li-Nux}
\date{Septembre 2012}

% Commandes personnalisées
\newcommand{\upcirc}{$^{\circ}$\xspace}
\newcommand{\super}[1]{$^{\mbox{\scriptsize #1}}$\xspace}

% Réduction des marges
\addtolength{\hoffset}{-1cm}
\addtolength{\textwidth}{2.5cm}
\addtolength{\voffset}{-2cm}
\addtolength{\textheight}{4.9cm}

% Logo du Louvain-Li-Nux
\newsavebox{\logollnux}
\sbox{\logollnux}{\raisebox{-2.5cm}{\includegraphics[height=2.5cm]{logo.png}}}

\begin{document}

% Titre

\begin{tabular}{p{12cm}r}

\begin{center}{\Large Les fiches du Louvain-li-Nux\linebreak \linebreak
\LARGE Episode 2: Ubuntu}\end{center}
&
\usebox{\logollnux}

\end{tabular}

% Contenu

\section*{Pourquoi Ubuntu est le choix idéal pour commencer?}



\section*{historique, étimologie, canonical}



\section*{nuances, LTS, Beta, 32 ou 64bit, Gnome Unity}



Pour plus d'informations:

\begin{center}\texttt{www.louvainlinux.be} \hspace{0.5cm}Rue Constantin Meunier 12 (Bruyères)\end{center}
                                                                                                                       
\end{document}	