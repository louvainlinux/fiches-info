\documentclass[11pt]{../fiche}

\begin{document}

\titlellnux{1}{Linux et les logiciels libres}

\section*{C'est quoi Linux?}

Linux est un système d'exploitation comme Windows et Mac OS. La première version fut publiée en 1991 par Linus Thorvalds.
Aujourd'hui, la majorité des superordinateurs ainsi que des serveurs web (ceux qui vous filent les sites quand vous surfez)
tournent sur linux. De nombreuses personnes l'utilisent sur leur PC et il se trouve également au coeur d'Android.
Linux est un système d'exploitation libre.
\vspace{0.5cm}

C'est bien bon dites vous, mais pourquoi utiliser linux sur son PC, qu'est-ce que ça a comme avantage ?\\
Cette liste non exthaustive contient quelques unes des raisons les plus souvent rencontrées:
\begin{itemize}
    \item C'est facile à utiliser et comporte une interface sympa et agréable (qui est personnalisable)
    \item Vous trouvez des milliers de programmes dans une bibliothèque d'applications (façon appstore) \textbf{gratuite}
    \item Pas de toolbars inutiles, plugins, publicités etc. dont personne n'a besoin
    \item Une fois les programmes installés, ils se mettent automatiquement à jour.
    \item Linux est très fiable. Vous n'aurez pas de virus (même pas besoin d'antivirus),
        pas de bluescreen, pas d'inquiétude!
    \item Linux fonctionne bien sur des vieux PC, petits netbooks et autres, et
        ne viellit pas: un vieux PC fonctionne aussi vite et bien qu'un neuf !
    \item Si vous avez un ordinateur très performant, Linux s'en servira efficacement
    \item Il y a plein de gens prêts à vous aider sur les forums internet.
    \item Pour ceux qui sont intéressés par apprendre l'informatique, on trouve
        facilement des explications sur le "comment ça marche" et on peut
        expérimenter.
    \item Et c'est gratuit, enfin légalement gratuit ;)
\end{itemize}

\section*{Et les logiciels libres?}

Un logiciel libre est caractérisé par plusieurs points :
\begin{itemize}
\item Il est libre d'utilisation: toute personne peut l'utiliser à toutes les fins possibles
\item Il est libre d'étude: son code source (recette de fabrication) est accessible par tout ceux qui veulent
\item Il est libre de modification: toute personne désirant ajouter un ingrédient dans la recette peut le faire
\item Il est libre de redistribution: toute personne peut recréer le même programme et le distribuer
\end{itemize}
Le logiciel libre s'oppose au logiciel propriétaire (ou privateur).

Une logiciel libre, tout comme un logiciel propriétaire, vient avec une
\textbf{license} d'utilisation (le long texte o\`u vous cliquez "j'accepte" sans
avoir lu...). Il s'agit d'un \textbf{contrat} entre l'utilisateur (vous !) et
le développeur/éditeur/vendeur/... Pour un logiciel propriétaire, il vous
impose des contraintes, tandis que pour un logiciel libre il vous garantit les
libertés citées ci-dessus.

La plupart des étudiants utilisent des logiciels libres, parfois sans le savoir.
Des exemples sont: Firefox, VLC, Gimp, Libre Office (ancien OpenOffice), jDownloader.
Les logiciels libres ne se limitent donc nullement aux ordinateurs sur Linux.

Des logiciels qui ne sont pas installés chez vous sur le disque dur mais
que vous accédez seulement au moyen d'un navigateur web sont, eux aussi, souvent libre.
Un exemple que vous conaissez tous est la plateforme moodle utilisé par l'UCL
ou encore les sites web construits avec Wordpress. Vous voyez donc que les logiciels
libres sont déployés à large échelle et fonctionnent de manière fiable.

\subsection*{Quelques avantages d'un logiciel libre}

Les logiciels libres ont de nombreux avantages.

Comme un logiciel libre est le résultat du travail de nombreuse personnes, il
en est d'autant plus efficace. Puisque chacun à la possibilité de corriger les
bugs, ceux-ci sont d'autant plus vite éliminés. S'il manque, une
fonctionnalité, celle-ci sera rapidement rajoutée. Et comme le nombre de
développeurs est souvent élevé et qu'ils ont tous des besoins spécifiques, cela
assure au programme un très grand nombre de fonctionnalités avancées et/ou un
grand nombre de programmes différent.

De plus, toujours grâce au grand nombre de développeurs potentiels,
les failles de sécurité sont corrigées très rapidement. Ce n'est pas
toujours le cas pour des logiciels propriétaires, pour lesquels il faut
faut attendre parfois plus de 6 mois: une aubaine pour les virus.

Ensuite, les logiciels libres sont généralement unis pour respecter des
standards. Ainsi, les logiciels vont essayer de former un ensemble cohérent,
partager des ressources et utiliser des formats de fichiers lisibles par
d'autre logiciels (cela permet de toujours conserver la lisibilité des données
même si le logiciel d'origine est obsolète, il n'y a pas de risque d'avoir des
données illisibles faute de logiciels adaptés).

Quatrièmement, la plupart des logiciels libres sont gratuits.

\subsection*{Le libre: pas que des logiciels}

Il y a plein d'autre oeuvres qui existent sous license libre: des films,
livres, musique, photos...

\section*{Et le Kot à projet Louvain-li-Nux alors?}
Notre projet est justement la promotion des logiciels libres et de Linux en particulier sur le site de Louvain-la-Neuve.
Pour ce faire, nous organisons entre autres une install party par quadri (ou on installe Linux sur plein d'ordis),
une séance d'introduction à \LaTeX{}, Git...

En plus, vous pouvez venir tous les lundis soirs chez nous et on va dépanner/améliorer votre ordinateur.

\end{document}
