\documentclass[12pt]{../fiche}

\usepackage{eurosym} 

\begin{document}

\titlellnux{15}{La neutralité du net, une inconnue... en danger}

\section{Qu'est-ce que c'est?}\label{kezako}

La neutralité du Net est un principe fondateur d'Internet qui garantit
que les opérateurs de télécoms ne discriminent pas les communications de
leurs utilisateurs, mais demeurent de simples transmetteurs
d'information. Ceci permet à tous les utilisateurs, quelles que
soient leurs ressources, d'accéder au même réseau dans son entier.

C'est un principe qui est aujourd'hui remis en cause à mesure que les
opérateurs développent des modèles économiques qui restreignent l'accès
à Internet de leurs abonnés, en bridant ou en bloquant l'accès à
certains contenus, services ou applications en ligne (protocoles, sites
web, etc.), ainsi qu'en limitant leur capacité de publication.

\section{Les arguments pour la neutralité du Net}

\paragraph{Démocratie et liberté d'expression}
Internet, ermet à chacun de publier du contenu avec une audiance
mondiale, facilement et à un coût assez faible.
Cela permet à chacun de s'exprimer publiquement, on peut donc dire qu'internet
est un moyen de mettre en oeuvre la liberté d'expression et donne à chaque
citoyen la possibilité de jouer le rôle de contre-pouvoir démocratique
(rôle traditionnellement attribué aux médias).

La neutralité du net garantit que le flux d'information vers les publications
de certaines personnes ne sont pas défavorisés parce que les auteurs n'ont pas
de grand moyens financiers et/ou gênent les autorités.

\paragraph{Innovation}
Sur internet, n'importe qui peut transmettre n'importe quel message
à quelqu'un d'autre, sans besoin d'autorisation.
Cela permet de créer des services innovants, très rapidement et avec
peu de moyens, par exemple des startups, des bénévoles (pensons à certains
logiciels libres).

La tendance actuelle est à controler tout les échanges, et à bloquer tout
ce qui est "inconnu"... il s'agit alors d'un frein à l'innovation.
Les nouveaux services peuvent aussi entrer en concurrence avec des
services existant (exemple: téléphone vs Skype), et leur blocage produit
des conditions de distortion de concurrence.

\section{Les arguments contre la neutralité du Net}

La neutralité du net peut parfois s'opposer aux mesures de \og{}qualité
de service\fg{}, qui augmentent la priorité de certains flux sur d'autres:
par example éviter les coupures sur les appels téléphoniques sur internet
au prix de ralentir des téléchargement.

Certaines de ces mesures sont légitimes, comme l'accès aux services d'urgence,
tandis que d'autres le sont moins: payer plus cher pour un accès prioritaire,
ralentir l'accès aux sites critiquant la politique du gouvernement...

\section{Exemples concrets de remise en cause de la neutralité du Net}

Les atteintes à la neutralité du réseau peuvent être le fait de
discriminations à l'égard de la source, de la destination ou du contenu
de l'information transmise via le réseau.

\subsection{Discrimination à l'égard de la
source ou destination}

En 2005, au Canada, l'opérateur Telus a bloqué l'accès vers des sites de
syndicats à l'occasion d'un mouvement social interne. Davantage encore
qu'une atteinte à la neutralité du Net, cette mesure fut dénoncée comme
de la censure.

En 2010, l'opérateur virtuel M6 Mobile utilisant le réseau Orange
annonce une offre à 1~\euro par mois ne donnant accès qu'aux pages web
des réseaux sociaux Facebook et Twitter.

En France, Orange a mis en place en aout 2010 des offres commerciales
internet mobile permettant, moyennant surcoût, d'accéder de façon
illimitée au service de musique en stream Deezer alors que son forfait
mobile est, pour les autres sources de contenus du même type, limité à 1
Go par mois, le rendant inutilisable pour accéder à des services
concurrents.

En France toujours, Bouygues Telecom a créé une offre garantissant un
accès \og{}prioritaire\fg{} sur le reste de ses clients en cas de congestion
du réseau.

\subsection{Discrimination à l'égard du
contenu}\label{discrimination-uxe0-luxe9gard-du-contenu}

En 2007, l'opérateur Comcast, qui possède également de nombreux groupes
de médias, a ralenti le trafic peer-to-peer sur ses réseaux.

En France, les opérateurs proposent des forfaits internet 3G+ qui
bloquent des services Voix sur réseau IP (tel que Skype). Le 13 avril
2010, Orange a annoncé l'autorisation des applications VoIP sur son
réseau, alors que SFR et Bouygues Telecom confirment leur volonté
d'offrir également l'accès à cette technologie.

\section{Pour aller plus loin}\label{pour-aller-plus-loin}


\begin{minipage}{0.65\textwidth}
    La \textbf{Quadrature du Net} est une association française qui
agit pour la défense des droits et libertés des citoyens sur Internet.

Ils campagnent:
\begin{itemize}
\item pour la défense de la neutralité du Net;
\item pour la protection du droit à la vie privée;
\item contre la surveillance de masse;
\item pour une réforme positive du droit d'auteur et contre la répression
du partage;
\item pour la défense de la liberté d'expression contre la censure;
\item pour l'exclusion des dispositions concernant nos droits fondamentaux
des accords internationaux.
\end{itemize}
\end{minipage}
\hfill
\begin{minipage}{0.25\textwidth}
\includegraphics[width=\linewidth]{quadraturenet.png}
\end{minipage}
\bigskip

\begin{minipage}{0.65\textwidth}
    La \textbf{fédération FDN} regroupe des Fournisseurs d'Accès à Internet
associatifs se reconnaissant dans des valeurs communes : bénévolat,
solidarité, fonctionnement démocratique et à but non lucratif; défense
et promotion de la neutralité du Net.
\end{minipage}
\hfill
\begin{minipage}{0.25\textwidth}
\includegraphics[width=\linewidth]{ffdn.png}
\end{minipage}

\section{Sources}\label{sources}

\url{fr.wikipedia.org/wiki/Neutralite_du_reseau}

\url{www.laquadrature.net}

\url{www.ffdn.org}

\end{document}
