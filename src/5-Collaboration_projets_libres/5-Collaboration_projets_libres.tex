\documentclass[10pt]{../fiche}

\begin{document}

% Titre
\titlellnux{5}{Le travail collaboratif}
\vspace{-.45cm}

% Contenu

\section*{Les soucis du travail en groupe}
\vspace{-.1cm}
Bien que nécessaire,
% dans l'environnement de travail au XXI\up{ième} siècle
le travail en groupe n'est pas toujours évident à mettre en \oe uvre.
Vous avez tous déjà rendu de tels travaux et vous connaissez certainement ces problèmes récurrents:
\begin{itemize}
  \setlength{\parskip}{0pt}
  \setlength{\parsep}{0pt}  
	% \item Il y a ces personnes surmotivées qui insistent à faire tout toutes seules, ou encore d'autres qui profitent de la première occasion pour se reposer. Mais même si tout le monde à la même motivation et connaissance, d'autres problèmes existent.
	% \item Certaines personnes avancent plus vite que d'autres
	\item Si chacun travaille dans son coin, vous vous retrouverez vite avec une pléthore de fichiers différents dans des formats tout aussi différents (\textit{doc}, \textit{docx}, \textit{odt}, \textit{txt}, \textit{xls}, \textit{pdf}, mails, etc.). Bien vite, vous ne saurez plus quelles versions prendre pour la mise en commun, comment fusionner ces changements et avec quels fichiers, etc.
	\item Séparer le travail en blocs complètement distincts pourrait aider, si c'est avant tout possible, mais le risque pourrait être que personne ne se sente responsable pour le texte en entier. Au pire, vous aurez un mélange de styles d'écriture différents et plein de redondances. Bref, pour avoir un rapport qui tient la route. Bonne chance !
	\item Si vous travaillez avec un outil comme \textit{Dropbox}, il est fréquent de voir des fichiers effacés, modifiés ou revenu dans une version antérieure sans réelle intention de la part d'un collaborateur tout en passant inaperçu puisque la dernière modification ou le dernier à se connecter à \textit{Dropbox} écrasera le travail des autres.
\end{itemize}
Que faire alors?

\vspace{-.3cm}
\section*{La gestion de projets de logiciels libres, un bon exemple ?}
\vspace{-.1cm}
Les programmeurs de logiciels libres sont souvent confrontés aux mêmes soucis. Parfois l'échelle est bien plus grande comme avec le projet du noyau Linux qui par exemple, lors des 7 dernières années, a vu collaborer pas moins de 8000 développeurs dont la plupart venant de 800 compagnies différentes de partout dans le monde. Allons voir comment ils se débrouillent et qu'est-ce que l'on peut en retirer !

\vspace{-.3cm}
\subsection*{Une hiérarchie, des responsables}
\vspace{-.1cm}
L'expérience montre que la mise en commun est une étape essentielle du travail en groupe. Pour des collaborations dans les logiciels libres où tout le monde peut participer, il est indispensable d'avoir au moins un responsable. Celui-ci est au courant des tâches des autres et s'assure que le projet suive son bon cours.
Dans de gros projets comme Linux, on peut remarquer une hiérarchie: il y a une personne à la tête, qui a le dernier mot en acceptant ou non des modifications et veille à la bonne réalisation du produit fini, suivi par une petite équipe qui pré-mâche son boulot en vérifiant que les changements sont corrects et enfin plusieurs équipes responsables d'une partie du programme.\\

Ce responsable s'occupe moins du travail concret (voire pas du tout si le nombre de collaborateurs est assez important). Il est à l'écoute des autres et il peut déléguer au cas où ; il n'est pas nécessaire d'avoir une connaissance du projet entier.
Un exemple à suivre pour les travaux de groupe~?

\vspace{-.3cm}
\subsection*{Des outils, des conseils}
\vspace{-.1cm}
De nombreux outils existent, beaucoup sont libres et la plupart sont gratuits.
Concernant les besoins spécifiques des projets de programmation de logiciels libres, il y a les gestionnaires de versions (\textit{GIT}, \textit{SVN}, \textit{BZR}, etc.) qui s'occupent de stocker le code source (la recette du programme). Étant donné que le code n'est au final que du texte, l'outil permet de facilement voir les différences entre les versions avec des commentaires, de facilement fusionner des parties de fichiers ou d'aider à gérer des conflits en cas de modification d'une même partie, et bien plus. Ainsi, un développeur peut ne partager que la partie du code qu'il a modifiée sans devoir, à chaque fois, tout ré-envoyer et passer du temps à vérifier qu'il est bien à jour.\\

On peut appliquer ceci à la conception de rapports en groupe. Tout d'abord, un bon conseil si vous êtes plusieurs: n'utilisez pas directement une suite bureautique (OpenOffice, LibreOffice ou Word) installée sur votre ordinateur. Préférez d'abord passer par des outils en ligne comme Google Document ou des \textit{pad} comme \url{FramaPad.org} (pour des réseaux internes, il y a aussi \textit{Gobby}). Ces outils permettent l'édition en ligne à plusieurs en même temps. On peut déjà s'occuper de la mise en page, c'est simple et peut-être limité mais c'est une bonne chose au début. Ensuite, il est possible d'exporter le document en plusieurs formats comme des \textit{PDF}, quand il ne faut plus rien modifier, ou des formats \textit{odt} ou \textit{doc} pour une édition ultérieur (si une fonctionnalité indispensable est manquante dans l'éditeur en ligne). À noter qu'il est également possible, avec ces outils, de récupérer des anciennes versions du document en cas de soucis.

Bien-sûr, on ne peut que vous conseiller l'utilisation de \LaTeX{} (et de son éditeur collaboratif Overleaf) qui permet de se focaliser sur le texte tout en aillant une mise en page impeccable. C'est par exemple très simple de séparer le document en plusieurs petits fichiers pour répartir le travail. \LaTeX{} est également sollicité par les mémorants et pour créer des rapports importants en taille. Cependant, il n'est pas facile à aborder mais le Louvain-Li-Nux organise à chaque début de quadri un atelier d'initiation.	Alors, si l'aventure te tente, suis notre page Facebook pour ne pas manger notre prochain atelier !

\end{document}
