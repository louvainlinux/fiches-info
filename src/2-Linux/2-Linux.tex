\documentclass[10pt]{../fiche}

\begin{document}

\titlellnux{2}{Linux, un système d'exploitation libre}

\vspace*{0.5cm} Dans notre 1\up{ière} fiche informative, nous avons vu ce que sont les logiciels libres. Cette semaine, nous allons te présenter notre petit chouchou : Linux.

\section*{C'est quoi Linux?}

Linux est un système d'exploitation (\textit{operating system} en anglais, abrévié OS), comme Windows ou MacOS. À la différence de ces derniers, Linux est libre. Si tu n'as jamais entendu parler d'un OS, en gros, c'est le chef d'orchestre de ton ordinateur : c'est lui qui gère tout ton matériel (clavier, écran, etc.) et tes logiciels.

La première version de Linux a été publiée en 1991 par Linus Thorvalds.
Aujourd'hui, la majorité des superordinateurs ainsi que des serveurs web (ceux qui te filent les sites quand tu surfes)
tournent sous Linux. De nombreuses personnes l'utilisent sur leur ordinateur et il se trouve également au coeur d'Android.
\vspace{0.5cm}

C'est bien beau dis-tu, mais pourquoi utiliser Linux sur son ordinateur ? Qu'est-ce que ça a comme avantages ?\\
Voici une liste non exthaustive des raisons les plus souvent rencontrées:
\begin{itemize}
    \item C'est facile à utiliser, avec une interface agréable et grandement personnalisable
    \item Tu y trouveras des milliers de programmes dans une bibliothèque d'applications (façon appstore) \textbf{gratuite}
    \item Pas de toolbars inutiles, plugins, publicités, etc. dont personne n'a besoin
    \item Une fois les programmes installés, ils se mettent à jour automatiquement
    \item Linux est très fiable: tu n'auras pas de virus,
        pas de bluescreen,... pas d'inquiétude!
    \item Linux fonctionne bien sur des vieux ordinateurs, petits netbooks et autres, et
        ne viellit pas : un vieil ordinateur fonctionne aussi vite et bien qu'un neuf !
    \item Si tu as un ordinateur très performant, Linux s'en servira efficacement.
    \item Il y a plein de gens prêts à t'aider sur les forums si tu as des questions
    \item Si l'informatique t'intéresse, tu trouveras
        facilement des explications sur le "comment ça marche" et tu pourras
        expérimenter
    \item Et c'est gratuit, enfin, légalement gratuit ;)
\end{itemize}



\section*{Et le Kot à projet Louvain-li-Nux alors?}
Comme tu le sais maintenant, notre projet est la promotion des logiciels libres en général, et de Linux en particulier, sur le campus de Louvain-la-Neuve.
Pour ce faire, nous organisons entre autres une install party par quadri, où on aide ceux qui le désirent à installer Linux sur leur ordi.

D'habitude, nous t'accueillons tous les lundis pour t'aider si tu as des problèmes avec ton Linux, ou si tu veux l'installer sur ton ordinateur. Cela n'est malheureusement pas possible en ce moment... Tu sais pourquoi. En attendant de pouvoir réouvrir nos permanences, nous pouvons t'aider en ligne, dans la mesure du possible. N'hésite pas à nous contacter via notre page Facebook Louvain-li-Nux !

A bientôt ;)

\end{document}