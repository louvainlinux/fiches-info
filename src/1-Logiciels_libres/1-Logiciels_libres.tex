\documentclass[11pt]{../fiche}

\begin{document}

% Titre
\titlellnux{1}{Les logiciels libres}

\vspace*{0.5cm} Au Louvain-li-Nux, notre projet est de promouvoir l'utilisation de logiciels libres. Tu ne sais pas ce qu'est un logiciel libre ? On est là pour te l'expliquer !

\section*{Qu'est-ce qu'un logiciel libre?}

Un logiciel libre (par opposition propriétaire) est caractérisé par quatre points :
\begin{itemize}
\item Il est libre d'utilisation : toute personne peut l'utiliser à toutes les fins possibles
\item Il est libre d'étude : son code source (recette de fabrication) est accessible par tous
\item Il est libre de modification : toute personne désirant ajouter un ingrédient dans la recette peut le faire
\item Il est libre de redistribution : toute personne peut recréer le même programme et le distribuer
\end{itemize}

Une logiciel libre, tout comme un logiciel propriétaire, vient avec une
\textbf{license} d'utilisation (le long texte où tu cliques "j'accepte" sans
avoir lu...). Il s'agit d'un \textbf{contrat} entre l'utilisateur (toi !) et
le développeur/éditeur/vendeur/... Pour un logiciel propriétaire, il
impose des contraintes, tandis que, pour un logiciel libre, il garantit les
libertés citées ci-dessus.
 Chacune de ces licences autorise un certain type d'utilisation. Certaines appliquent le principe du \textit{Copyleft} selon lequel tout dérivé d'une oeuvre doit rester sous la même licence ; cela permet de préserver le coté libre d'une œuvre.
 
 Nous nous focalisons ici sur les \textit{logiciels} libres, mais les licenses libres peuvent s'appliquer à tout type d'oeuvre, comme de la musique, des films, des livres, des images,...

\section*{Quelques avantages d'un logiciel libre}

Tout d'abord, les logiciels libres sont souvent le résultat du travail de nombreuses personnes ; ils en sont d'autant plus efficaces. Chaque personne contribuant au programme ayant ses propres besoins, s'il manque une fonctionnalité, elle sera sûrement rapidement ajoutée. Cela assure aux programmes un très grand nombre de fonctionnalités avancées.

De plus, toujours grâce au grand nombre de développeurs potentiels,
les failles de sécurité et autres bugs sont corrigées très rapidement. Et cela n'est pas
toujours le cas pour des logiciels propriétaires, pour lesquels il 
faut attendre parfois plus de 6 mois : une aubaine pour les virus.

Ensuite, les logiciels libres sont généralement unis pour respecter des
standards. Ainsi, les logiciels vont essayer de former un ensemble cohérent,
partager des ressources et utiliser des formats de fichier lisibles par
d'autres logiciels (cela permet de toujours conserver la lisibilité des données
même si le logiciel d'origine est obsolète, il n'y a pas de risque d'avoir des
données illisibles faute de logiciel adapté).

Enfin, la plupart des logiciels libres sont gratuits.

\section*{Quelques exemples de logiciels libres}

Ça t'a donné envie de découvrir des logiciels libres ? A vrai dire, tu en utilises sûrement déjà, peut-être sans le savoir ;)

Par exemple, Firefox est un navigateur libre. D'ailleurs, beaucoup de logiciels qui sont accessibles depuis un navigateur sont libres. C'est le cas, notamment, de la plateforme Moodle, ou encore des sites web construits avec Wordpress.

Pour consulter tes e-mails, au lieu de Outlook, tu peux utiliser \textit{Mozilla Thunderbird} et pour remplacer la suite \textit{Microsoft office}, la suite \textit{LibreOffice}.

Come lecteur vidéo, on te propose \textit{VLC}, et pour gérer ta musique, \textit{Clementine}.

Pour lire des fichiers PDF, et bien d'autres formats, tu peux utiliser \textit{Sumatra} (pour \textit{Windows}) ou \textit{Evince} (pour \textit{Linux})

Si tu es amateur de montage photo, pas besoin de dépenser ton salaire d'étudiant dans une licence \textit{Photoshop}, tu peux utiliser Gimp (un tuto sera bientôt disponible sur notre site !). Et si tu aimes la modélisation 3D, \textit{Blender} est là pour toi.

Si tu es plutôt musicien, tu trouveras sans doute ton bonheur avec \textit{Ardour}, \textit{LLMS}, \textit{Audacity}, \textit{RoseGarden}, \textit{TuxGuitar} ou encore  \textit{Hydrogen}.

Pour gérer tes fichiers compressés, tu peux utiliser \textit{7-zip}.

Enfin, à la place de \textit{Matlab}, tu peux utiliser \textit{Octave}.

\vspace*{0.5cm}Tu en veux plus ? Sur \url{https://framalibre.org/}, tu trouveras une liste
de milliers de logiciels libres classés par domaine d'application. N'hésite pas également à venir discuter avec nous pour trouver le logiciel dont tu as besoin ! Si tu veux savoir comment installer ces logiciels, tu trouveras sans doute facilement l'information en ligne, sinon, tu peux aussi nous contacter et on t'aidera avec plaisir.

A la semaine prochaine pour une nouvelle fiche info !

\end{document}
