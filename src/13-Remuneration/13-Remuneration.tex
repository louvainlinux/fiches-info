\documentclass[12pt]{../fiche}

\begin{document}

\titlellnux{13}{Vhere Is Da Monheyyy ???}
Salut à vous futur libriste ! (à ne pas confondre avec libraire) 
\vspace{0.8em}

Bon, maintenant que vous savez ce que c'est le libre (voir la fiche info 11, normalement elle doit être acrochée quelque part dans vos toilettes. Sinon il y a un rappel en bas de page\footnote{Le libre, c'est une philosophie centrée autour du partage, et du fait de permettre a tous de remodifier et repartager les oeuvres de n'importe qui, du moment que l'auteur est crédité.}), vous vous demandez sûrement comment les projets libre font pour gagner de l'argent… Ben oui, si n'importe qui a accès à la manière dont vous avez fait votre oeuvre, on peut facilement la refaire gratuitement.
\vspace{0.8em}

Pour financer des projets libre, la méthode la plus répandue est le financement par don; ce sont les utilisateurs qui donnent de l'argent quand ils le désirent, ou alors des sociétés qui ont un intéret a voir tel ou tel projet avancer plus vite (ben oui, tout va plus vite quand on a la thu-thune). C'est par exemple le cas de Wikipedia (sauveur des étudiants depuis 2001) ou de la Linux Fundation, qui est sponsorisée par Microsoft.
\vspace{0.8em}

Une autre solution, surtout utilisée dans les domaines artisitiques, est de vendre ses oeuvres à prix libre; lors de l'achat/téléchargement, il sera demandé a la personne combien elle désire payer, et elle peut mettre un certain montant en fonction de ses moyens et de son appréciation de l'oeuvre. Il est également possible de ne rien payer, puisque que le prix est libre ! (mais c'est pas super gentil ! :P )
\vspace{0.8em}

Le dernier moyen fort répandu --- en particulier dans le cas des logiciels libre --- est le payement du support technique.
Par exemple, Canonical, qui est la société qui supervise le développement d'Ubuntu, fournit Ubuntu gratuitement, mais fait payer le service technique (par exemple à des entreprises ayant des besoins particuliers). Cela permet de fournir un service gratuit, tout en le finançant à travers un service payant optionnel.
\vspace{0.8em}

Voilà, vous savez maintenant d'ou vient le pognon du libre ! Il s'agit des plus grandes méthodes de financements, il en existe d'autres, moins répandues que celles-ci.

\end{document}
