\documentclass{../fiche}

\begin{document}

% Titre
\titlellnux{17}{Open Science}
\paragraph{}
Ami du libre, ami de la science, bonjour!
Si tu n'es pas en master ou que tu ne t'es pas encore confronté aux publications scientifiques, le concept d’Open Science peut te paraître obscur.
Le principe de base énonce que chacun devrait pouvoir avoir accès librement et gratuitement aux résultats de la recherche (publications, données, ...) et aux ressources éducatives.

\section{Open Access}
Afin de faire profiter à la communauté scientifique de leurs résultats, autrement dit, pour faire avancer la science, les chercheurs doivent publier leurs résultats. Je te passe les détails du processus, car il est assez compliqué. Ce qu’il faut retenir, c’est que lorsqu’un chercheur ou un étudiant souhaite lire cet article, il doit le payer. C’est normal, après tout il faut financer les personnes responsables de l’édition et de la vérification du contenu scientifique. Néanmoins, s’abonner à toutes les revues pertinentes pour ton domaine d’étude est vite une chose impossible, vu le coût prohibitif des abonnements (200 euros pour un journal comme Nature). Fort heureusement, la plupart des universités disposent d'abonnements à ces journaux. Ainsi, tu peux consulter gratuitement un grand nombre d'article via le proxy UCL chez les éditeurs. Malheureusement, ces abonnements coûtent très cher à l'université et représentent une grande partie du budget des bibliothèques. De plus, si tu es mémorant, tu dois sans doute être souvent frustré de ne pas avoir accès à un article si l’UCL n’est pas abonnée à la revue. L’Open Access répond à cette problématique : en participant aux frais de publications (entre 1000 et 3000 euros), le chercheur met à disposition gratuitement son article aux lecteurs. Ce mode de publication permet de toucher un plus grand nombre de lecteurs, et permet aux institutions qui n’ont pas de gros budgets d’accéder aux derniers résultats scientifiques (on pense notamment aux pays en voie de développement).


\section{Open Data}
Dans un monde où le nombre de données récoltées croît à une vitesse exponentielle, il est important de pouvoir en tirer quelque chose de pertinent. Mais sais-tu que la plupart des données récoltées sont enfouies dans des disques durs? En effet, il n'y a pas assez de personnes et de ressources pour toutes les analyser. Dès lors, pourquoi ne pas mettre à disposition du public certaines de ces données? C'est ce que fait déjà Kaggle, qui propose des compétitions d'analyse de données. Si tu as envie de faire des découvertes inédites en analysant une base de données de pokémon, des films sur Imdb ou encore sur la criminalité dans les villes, je te conseille d'aller faire un tour sur ce site web. Plus sérieusement, peu de chercheurs mettent à disposition du public les données issues de leurs expériences. C'est dommage, car ça permettrait à d'autres chercheurs de vérifier la validité des résultats, voir de proposer d'autres méthodes d'analyses et qui sait, peut être faire une découverte majeure! Il existe toutefois une initiative en Wallonie : l’Agence du Numérique met à disposition de tout le monde des données ouvertes issus des services publics.

\section{Ressources éducatives libres}
Tu en utilises souvent une sans forcément t'en rendre compte: Wikipédia! Le but de ce mouvement est de mettre à disposition de tous et gratuitement des supports de cours, vidéos, ... Les MOOCS en sont un bon exemple (edX, Coursera). 

\section{Conférence Open Science}
Tu souhaites en apprendre plus ? Alors viens à notre conférence sur l’Open science ce 9 mai 2018 à 16h30 à l’Agora 4.

\end{document}
