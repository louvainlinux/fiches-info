\documentclass[12pt]{../fiche}

\usepackage{siunitx}
\usepackage{eurosym}

\begin{document}

\titlellnux{9}{Petite fiche d'une prof de français}


Tu penses que le logiciels libres, c'est-à-dire le projet du Louvain-li-Nux, c'est que pour les geeks - sous-entendez les sinfs ou ingé civils ? Et bien tu as perdu !! (Moi aussi d'ailleurs, mais on ne va pas tout mélanger!)

Des logiciels libres, il en existe dans tous les domaines, et pour tous les programmes: Linux est l'équivalent de Windows, Firefox pour Internet Explorer, Gimp pour Photoshop, Thunderbird pour Outlook…

Et comme tu l'as vu dans le titre, cette fiche va s'attaquer à l'ORTHOGRAPHE ! Parce que oui, mécréant massacreur de la langue française dans tes mails et travaux, les logiciels libres - par l'intermédiaire des geeks du Linux - viennent à ton secours, et de manière gratuite !

Tout d'abord, un peu de vocabulaire. Correcteur (ou correcticiel en canadien) comprend généralement plusieurs ``types de correcteurs'' en même temps. Le plus connu étant les correcteurs orthographiques et grammaticaux, mais il existe aussi des correcteurs stylistiques (qui corrige ton style d'écriture, les répétitions...), et bien d'autres.

Tu utilises sans doute le correcteur orthographique et grammatical de base intégré dans Microsoft Word (ou Libre Office si t'es un type bien), et c'est le minimum. Il te sauve de quelques coquilles – et encore – mais il est loin de tout corriger ! Bien qu'il existe des correcteurs bien meilleurs (on te les donnera plus bas), de manière générale, aucun correcteur n'est infaillible. Autrement dit, ce super petit programme ne connaîtra jamais toutes les subtilités de la langue française et tu devras donc toujours vérifier après lui s'il ne laisse pas de fautes voire même s'il n'en a pas créées de nouvelles.

Le correcteur le plus connu et le plus complet du moment est ``Antidote'', parfois aussi appelé ``Druide'' et développé par les Canadiens (forcément). Ne te rue pas sur ton ordi pour autant, la licence est à plus de \si{100}{\euro}, bref impayable pour les étudiants. Heureusement, il existe de nombreux autres logiciels qui peuvent t'aider, sans aucun coût! Je les ai tous testés pour toi et te transmets les plus performants!

Certains correcteurs libres nécessitent de copier-coller ton texte sur le site (et donc de perdre ta mise en page) tels que \url{http://bonpatron.com/}, \url{http://www.cordial-enligne.fr/}.
Tandis que d'autres sont ``intégrés au logiciel de traitement de texte'' et donc corrigent directement dans ton document, notamment (mais pas le meilleur) \url{http://www.scribens.fr/}.

Comme on aime garder le meilleur pour la fin, on va, a présent, parler de Grammalecte (\url{www.dicollecte.org/}). Ce logiciel est considéré comme le meilleur correcteur et sait s’intégrer à tout: Word,
open office, Firefox, chrome,...

Pas si vite ! Grammalecte ne fais pas que correcteur aurtaugrafik ($\leftarrow$ rien que ce mot justifie l’utilité d’un correcteur) ! Grammalecte fait également correcteur de style, de conjugaison et à un très bon Bescherelle intégré. Si tu n’es pas sûr si c’est ca, sa ou ça, demande-le à grammalecte et tu auras ta réponse dans la seconde.

Voili voilà, on espère que maintenant on ne verra plus aucune faute dans les mails que tu nous enverras nous envoyer à l'avenir! ;-)


Kissous libres et gratuits!


Pierre du Linux


P.S.: Nous déclinons toute responsabilité en cas de fautes dans un travail passé sous correcteur. Comme précisé plus haut, un correcteur informatique n'est jamais aussi efficace qu'un correcteur humain. Pour tous les cas importants, passer un correcteur ne suffit pas, fais-le également relire par une personne maîtrisant bien les subtilités orthographiques du français!


\end{document}
