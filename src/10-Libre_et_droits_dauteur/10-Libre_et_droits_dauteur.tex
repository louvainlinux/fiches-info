\documentclass[12pt]{../fiche}

\begin{document}

\titlellnux{10}{Libre et droits d'auteur}


Le copyleft est l'autorisation donnée par l'auteur d'un travail soumis au droit d'auteur (œuvre d'art, texte, programme informatique ou autre) d'utiliser, d'étudier, de modifier et de diffuser son œuvre, dans la mesure où cette même autorisation reste préservée.

L'auteur refuse donc que l'évolution possible de son travail soit accompagnée d'une restriction du droit à la copie, à l'étude, ou à de nouvelles évolutions. De ce fait, le contributeur apportant une modification (correction, ajout, réutilisation, etc.) est contraint de redistribuer ses propres contributions avec les mêmes libertés que l'original. Autrement dit, les nouvelles créations réalisées à partir d'œuvres sous copyleft héritent de fait de ce statut de copyleft : ainsi, ce type de licence permet un partage de la création ou de la connaissance, comme bien commun, qui permet aux œuvres culturelles d'être développées librement.


\end{document}
