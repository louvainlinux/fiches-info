\documentclass[10pt]{../fiche}

\begin{document}

\titlellnux{3}{Au delà des logiciels}

\vspace*{0.5cm} 
Comme rapidement évoqué dans notre première fiche info, le monde du libre ne se limite aucunement aux logiciels ni même à l'informatique en général. Cette semaine, nous allons parcourir quelques exemples d'autres domaines où l'on trouve du libre.

\paragraph{Wikimedia}
Tu connais bien sûr Wikipédia : c'est LA référence pour toutes les recherches rapides d’information. La richesse de cette encyclopédie est que tout le monde peut y contribuer, même toi ! Elle est d'ailleurs parois controversée à cause de l'anonymat de ses contributeurs; notons toutefois que si les contributeurs sont anonymes, ils doivent cependant citer leurs sources. Cette encyclopédie libre a été fondée par \textsc{Jimmy Wales} en 2001. Deux ans après, il fonde la \textit{Wikimedia Foundation} qui est l'organisation qui gère, non seulement Wikipédia, mais aussi tous les autres projets Wiki- tels que Wiktionnaire, Wikibooks, Wikinews et beaucoup d'autres. 

\paragraph{Open Street Map}
Le principe d'\textit{Open Street Map} est le même que pour \textit{Wikipedia} : tout le monde peut aider à y construire les cartes. Grâce à cela, ces cartes sont généralement beaucoup plus détaillées qu'une carte papier, ou encore que celles de \textit{Google Maps} ! (Vas-y, fais le test, tu verras...)

\paragraph{Musique, livres et autres \oe uvres}
Il y a deux genres de musique et livres libres : ceux qui ont été publiés sous license libre, et ceux qui sont tombés dans le domaine publique.

Pour le premier cas, il existe les \textit{Creative Commons} (\textit{CC}) qui sont des licences grâce auxquelles un auteur (de musique ou autre) peut permettre
d'utiliser son \oe uvre tant que certaines conditions sont remplies. Par exemple, la mention \texttt{BY} veut dire qu'il faut mentionner l'auteur original, \texttt{NC} permet l'utilisation de l'\oe uvre
uniquement dans des circonstances non commerciales et \texttt{SA} permet de la réutiliser tant que le nouveau produit reste sous la même licence libre (principe du \textit{copyleft} dont on a parlé dans l'épisode 1). Tu trouveras plein de musiques libres sur \url{www.jamendo.com}. Pour les livres, mentionnons à titre d'exemple les romans policiers et de science-fiction de \textsc{Cory Doctorow}. On te recommande également les courts-métrages réalisés avec \textit{Blender} disponibles sur \url{www.blender.org/about/projects/} et particulièrement \textit{Sintel}, une histoire de dragons !

Pour le second cas, il s'agit d'\oe uvres dont le copyright a expiré. Actuellement, c'est le cas 70 ans après la mort de l'auteur.
%Il n'est pas très utile de retenir ce chiffre puisqu'il se fait scandaleusement prolonger à chaque fois que des \oe uvres comme \textit{Micky Mouse}
%ou les chansons d'\textit{Elvis} risquent de tomber dans le domaine public.
La plupart du patrimoine culturel qu'est la musique classique est aujourd'hui libre de droits d'auteurs.
%C'est entre autres pour cette raison qu'on en entend autant dans la publicité et les films.
Attention, c'est uniquement l'ensemble des notes de musique et non l'interprétation par un orchestre particulier qui est libre.
Le site web \url{www.musopen.org} présente un catalogue de musique classique rejoué avec l'intention de rendre ces interprétations libres. Pour les lives, on trouve beaucoup de scans de livres historiques.
%Nous vous conseillons de les télécharger dans le format \texttt{flac} (\textit{free lossless audio codec}) en raison de la qualité supérieure aux \texttt{mp3}.

Sur le site \url{archive.org} vous trouvez une grande collection de ces \oe uvres libres, dont plus de 500.000 films, un million d’enregistrements audio et trois millions de livres.
Si tu veux, par exemple, revoir comment la presse a réagi lors des événements du 11 septembre, tu le trouveras ici.
%Pour alléger le trafic sur leurs serveurs, l'archive fait usage du protocole \textit{Torrent}, souvent maladroitement lié aux échanges illégaux de fichiers.

\paragraph{Des cours}
Ceci est particulièrement intéressant pour les étudiants: presque toutes les universités de renommée mondiale publient en ligne une partie de leurs cours et conférences.
Ces vidéos sont typiquement sous licence \textit{CC}. \textit{Harvard, Oxford, MIT, Yale, Princeton, HEC Paris, Stanford, }... Tous les grands noms y sont.
La distribution peut se fait sur le site web de l'université, au moyen de flux \texttt{RSS}, dans \textit{iTunes}, \textit{Youtube}, etc.
En ce qui concerne l'UCLouvain, elle s'est lancée depuis 2013 dans la moocXperience, projet visant à créer des cours donnés exclusivement en ligne, accessibles à tous.
%Notons également que le professeur O. \textsc{Bonaventure} a reçu récemment un prix américain pour la publication de syllabus sous licence libre. %% mentionner le syllabus libre de O. Bonaventure?

%\paragraph{\textit{Internet archive}}
%\textit{Internet archive} s'est aussi donné le but d'archiver le web. Le \textit{Wayback machine} permet ainsi de consulter des sites web tels qu'ils étaient à des moments précis dans le passé.

\paragraph{Open data} Sur \url{www.kaggle.com}, des bases de données libres sont mises à la disposition de tout le monde. Le principe ? Permettre à chacun de réaliser des analyses statistiques sur ces données, mais aussi de modifier/améliorer les analyses réalisées par d'autres utilisateurs. Ainsi, cela permet une meilleure compréhension des données et une validation des analyses d'autrui. Fini les manipulations de données ou les analyses obscures! Chacun peut vérifier ce que l'autre a fait et le critiquer. De plus, cela permet aux étudiants en statistique/sciences des données de s'entraîner et apprendre des autres !

\end{document}
