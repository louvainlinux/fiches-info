\documentclass[12pt]{../fiche}

\begin{document}

\titlellnux{12}{L'art libre, les licences et moi}
Salut à toi futur ou actuel ami du Libre ! 
\vspace{0.8em}

Tu penses que le libre ne peut s'appliquer qu'à des logiciels, et que créer quelque chose de libre implique d'abandonner ses droit sur son travail?\\
Détrompe toi ! Le libre peut également s'appliquer à l'art, et il existe de nombreuses licences pour protéger son travail, tout en le gardant libre !
\vspace{0.8em}

L’art Libre regroupe toute forme d’art distribuée au plus grand nombre de personnes sans coût direct. Ceci regroupe donc les spectacles de rue, les graffitis, l’art distribué par internet et bien d’autres choses encore ! L’art Libre est une notion qui est apparue suite à la création du logiciel libre en 1984. Ce sont les responsables du langage de programmation Perl qui inaugurent l' Artistic License (licence artistique). Il existe plusieurs formes d’art libre qui sont apparues autour de la même période (BD, littérature, films, musique, ...).
\footnote{Pour plus d'info, la bonne vieille valeur sûre, https://fr.wikipedia.org/wiki/Art\_libre}
\vspace{0.8em}

C'est bien cool de faire de l'art libre, mais comment faire pour être sûr, par exemple, que les gens vont bien vous créditer lorsqu'il vont publier leur "remix tr0p d4rk feat. DJeuns"? Et bien grâce aux licences ! Elle permettent d'indiquer à quel point une oeuvre est libre, et ce que les gens peuvent faire avec.
\vspace{0.8em}

La licence creative commons est l’une des licence les plus connue de part sa flexibilité.
En effet, cette licence permet de définir des niveaux de restrictions, et permet ainsi à chaque auteur de choisir à quel point il veut protéger son travail; Il y a la licence creative commons de base (BY), qui demande juste de créditer l’auteur, la licence sharalike (BY-SA) qui demande à ce qu’en plus de créditer l’auteur, les oeuvres modifées gardent aussi la même licence (aussi appelée copyleft). Il existe également les variantes interdisant l’usage commercial et interdisant également les dérivées, et n’étant du coup plus des licences libres.
\vspace{0.8em}

La licence art libre est une licence très proche du copyleft, spécialement prévue pour l’art. 
Elle permet l’usage libre de la création, mais impose de garder libre toutes les créations en découlant (remix, retravail de l’image, etc.), favorisant ainsi la création de contenu libre. Une société ne sera donc pas en mesure de garder secret un travail fait sur base d'un copyleft. C’est une situation gagnant-gagnant ; l’entreprise peut utiliser tout le travail libre déjà effectué, mais doit également contribuer de son côté.
\vspace{0.8em}

Voila, vous en savez maintenant plus sur l'art libre ainsi que les licences ! Et la prochaine fois que vous ferrez un copier coller de wikipedia, penser à créditer les auteurs :p

\end{document}
