 
\documentclass[12pt]{../fiche}
\usepackage{hyperref}

\begin{document}

	
	\titlellnux{17}{Framasoft: Dégooglisons internet!}
	
	\vspace{0.8em}
	
	Salut à toi, futur ou actuel ami du Libre ! 
	
	\vspace{0.8em}
	
Nous vivons dans un monde numérique dominé par 5 grandes entreprises: Google, Apple, Facebook, Amazon et Microsoft. On les appelle aussi les GAFAM et celles-ci sont omniprésentes dans notre vie quotidienne. Grâce à Facebook, tu peux rester en contact avec tes amis. Chercher une information sur internet ? Google semble ton ami. Quant à Amazon, il te permet d'acheter sans quitter ton foyer. Enfin, tu as surement Windows sur ton PC, qui est produit par Microsoft. Ce sont les services les plus connus de ces entreprises, et elles ont un quasi monopole sur ceux-ci. 

Néanmoins, ces entreprises proposent beaucoup d'autres services, issus de leur politique agressive de rachat d'entreprises. Voici une petite liste non exhaustive de leurs acquisitions, mais vous pouvez consulter la liste complète sur Wikipédia.
\begin{itemize}
	\item Google à racheté Waze, Youtube, Android, Motorola, Nest et Deepmind
	\item Amazon à racheté Audible, Twitch
	\item Facebook à racheté Whatsapp, Oculus VR et Instagram
	\item Apple a racheté Shazam, Beats, Siri
	\item Microsoft à racheté Skype, LinkedIn et Nokia mobile
	
\end{itemize}

En bref, on peut voir que ces entreprises contrôlent une grande partie de notre vie numérique. Bien que tous ces services (et bien souvent gratuits!) soient pratiques dans notre vie de tous les jours, ceci à un prix: ces entreprises exploitent nos données personnelles afin de nous proposer des publicités toujours plus ciblées. De plus, la mainmise qu'elles ont sur le monde numérique à de quoi faire peur... Heureusement, le libre et les logiciels open source peuvent apporter une réponse à cette problématique!




	\section*{Framasoft}
	Framasoft est une ASBL française, qui a pour but de promouvoir le libre dans tous ses aspects, autant physiques (livres libres), que numériques. Elle est issue du milieu éducatif, et n'est pas majoritairement composée d'informaticiens. Elle milite pour un monde numérique émancipateur, et se positionne comme un trait d'union entre la communauté des libristes et le grand public.
	
	Créée à la base pour répondre au monopole de Microsoft avec Windows dans les année 2000, Framasoft s'est récemment inquiétée de ce nouveau contexte de monopole de la vie numérique par les GAFAM. Ainsi, elle a lancé la campagne "Dégooglisons internet!", qui propose une trentaine de services alternatifs éthiques. Framasoft s'engage à ne pas enregistrer et utiliser tes données. Leurs services sont décentralisés: ils sont hébergés par Framasoft et plusieurs personnes indépendantes (les CHATONS), bien souvent militantes dans le milieu libre.
	
	Voici quelques exemples de leurs services en ligne: Framagenda, Framaforms (pour remplacer Google forms), Framadate (Doodle), Framaboard (Trello), Framatalk (Skype), Framadrop (partage de fichiers)... Pour plus d'informations : \url{https://degooglisons-internet.org/fr/alternatives}
	
	
	
	Enfin, ces services libres se reposent sur le principe de solidarité afin d'assurer leur pérennité, à travers les dons et contributions au code source.



	\section*{D'autres alternatives libres}
		Framasoft publie également une liste de logiciels alternatifs libres. Tu peux la consulter sur: \newline \url{https://framalibre.org/alternatives}. 

	
	
\end{document}
