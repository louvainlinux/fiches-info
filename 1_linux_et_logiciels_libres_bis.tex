%%%%%%%%%%%%%%%%%%%%%%%%%%%%%%%%%%%
% Header pour fiche info du LLNux %
%     Version 1.1 (03/11/2012)    %
%%%%%%%%%%%%%%%%%%%%%%%%%%%%%%%%%%%

\documentclass[a4paper,10pt]{article}

\usepackage[utf8]{inputenc}           % UTF-8
\usepackage[a4paper, dvips]{geometry} % mise en page générale
\usepackage{lmodern}
\usepackage[T1]{fontenc}              % polices
\usepackage{multicol}                 % multi-colonnes avancé
\usepackage{xspace}                   % espaces conditionnels
\usepackage{graphicx}                 % images
\usepackage[francais]{babel}          % français

\usepackage{amsmath}                  % math
\usepackage{amssymb}                  % math
\usepackage{gensymb}                  % math

\usepackage{url}
	\urlstyle{sf}
\usepackage[usenames]{color}
\usepackage{varioref}                 % \vpageref

\usepackage{fancyhdr}                 % cfoot
\pagestyle{fancy}                     % we need info at the bottom but not a footnote
\renewcommand{\headrulewidth}{0pt}    % ça crée une ligne au-dessus... on l'enlève
% \thispagestyle{empty}               % we need only one page

\renewcommand{\familydefault}{\sfdefault}

\definecolor{codeBlue}{rgb}{0,0,1}
\definecolor{webred}{rgb}{0.5,0,0}
\definecolor{codeGreen}{rgb}{0,0.5,0}
\definecolor{codeGrey}{rgb}{0.6,0.6,0.6}
\definecolor{webdarkblue}{rgb}{0,0,0.4}
\definecolor{webgreen}{rgb}{0,0.3,0}
\definecolor{webblue}{rgb}{0,0,0.8}
\definecolor{orange}{rgb}{0.7,0.1,0.1}

% Configuration de babel
\NoAutoSpaceBeforeFDP

% Configuration de graphicx
\graphicspath{{images/}}

% Titre, auteur et date
\title{Informations sur les logiciels libres}
\author{Kot-à-projet Louvain-Li-Nux}
\date{\today}

% Commandes personnalisées
\newcommand{\upcirc}{$^{\circ}$\xspace}
\newcommand{\super}[1]{$^{\mbox{\scriptsize #1}}$\xspace}

% Réduction des marges
% Dimensions de la page :       	

  %%%%%%%%%%%%%%%%%%%%%%%%%%%%%%%%%%%%%%%%  0
  %   |                                  %
  %---+----------------------------------%  1
  %   | +----------------------------+   %  2
  %   | |          en-tête           |   %
  %   | +----------------------------+   %  3
  %   | +----------------------------+   %  4
  %   | |                            |   %       Remarques : 
  %   | |                            |   %        . distance de '0' à '1' : un pouce + \voffset
  %   | |                            |   %        . distance de 'a' à 'b' : un pouce + \hoffset
  %   | |           texte            |   %
  %   | |                            |   %
  %   | |                            |   %
  %   | |                            |   %
  %   | +----------------------------+   %  5
  %   | +----------------------------+   %
  %   | |         bas de page        |   %
  %   | +----------------------------+   %  6
  %%%%%%%%%%%%%%%%%%%%%%%%%%%%%%%%%%%%%%%%
  %a  b c                            d

   % général
     \voffset       -40mm    % pour descendre (si positif) ou remonter (si négatif) le tout
     \hoffset       -12.5mm    % pour agrandir (si positif) ou diminuer (si négatif) la marge gauche
     \oddsidemargin 10pt   % distance de 'b' à 'c'
   % texte
     \headsep       42pt   % distance de '3' à '4', la distance entre l'en-tête et le texte
     \textheight    760pt  % distance de '4' à '5', pour déterminer la hauteur du texte
     \textwidth     177.5mm  % distance de 'c' à 'd' 
   % en-tête
     \topmargin     0pt    % distance de '1' à '2', pour descendre (si positif) ou remonter (si négatif) le tout
     \headheight    14pt   % distance de '2' à '3', doit être > 13.59999
   % bas de page
     \footskip      38pt   % distance de '5' à '6', la distance entre le texte et le bas de page
%\addtolength{\hoffset}{-1cm}
%\addtolength{\textwidth}{2.5cm}
%\addtolength{\voffset}{-2cm}
%\addtolength{\textheight}{4.9cm}

% Logo du Louvain-Li-Nux
\newsavebox{\logollnux}
\sbox{\logollnux}{\raisebox{-2.2cm}{\includegraphics[height=2.5cm]{../logo.png}}}
\newsavebox{\logoqr}
\sbox{\logoqr}{\raisebox{-2.2cm}{\includegraphics[height=2.5cm]{../qr.png}}}

%titre
\newcommand{\titlellnux}[1]
{
	\hspace{-1cm}
	\begin{tabular}{lp{13cm}r}
		\usebox{\logoqr}
			&
		\begin{center}{\Large Les fiches du Louvain-li-Nux\linebreak \linebreak
		\LARGE \textbf{Episode #1}}\end{center}
			&
		\usebox{\logollnux}
	\end{tabular}
	\vspace{-.5cm}
}

% pied de page
\rfoot{\url{www.louvainlinux.be}\hspace{-3.0cm}}
\cfoot{}
\lfoot{Rue Constantin Meunier 12 (Bruyères)}

% PDF
   \usepackage[pdftitle={Fiche Informative},  % apparition ds les propriétés du doc
               pdfauthor={Louvain-li-Nux},
               pdfsubject={Fiche Informative du Louvain-li-Nux},
               pdfkeywords={louvain-li-nux,llnux,fiche info,lln,kap,ucl},
	       colorlinks=true,
	       linkcolor=webdarkblue, 
	       filecolor=webblue, 
	       urlcolor=webdarkblue,
	       citecolor=webgreen]{hyperref}


\begin{document}

% Titre
\begin{tabular}{p{13cm}r}
	\begin{center}{\Large Les fiches du Louvain-li-Nux\linebreak \linebreak
	\LARGE Episode 1: Linux et les logiciels libres}\end{center}
		&
	\usebox{\logollnux}
\end{tabular}

% Contenu

\section*{C'est quoi Linux?}

Linux est un système d'exploitation comme Windows et Mac OS. La première version fut publiée en 1991 par Linus Thorvalds.
Au jour d'hui, la majorité des Superordinateurs ainsi que des serveurs web (ceux qui vous filent les sites quand vous surfez)
tournent sur linux. Linux est un système d'exploitation libre.
\vspace{0.5cm}

C'est bien bon dites vous, mais à quoi est-ce que ça me sert concrètement?\\
Cette liste non exthaustive contient quelques unes des raisons les plus souvent rencontrées:
\begin{itemize}
\item C'est facile à utiliser et joli (ma grand mère sait le piloter)
\item Vous trouvez vos programmes dans une espèce d'appstore \textbf{gratuit}
\item Pas de bêtes toolbars, plugins, publicités etc. dont personne n'a besoin
\item Une fois les programmes installés, ils se mettent automatiquement à jour
\item Linux est très fiable: pas de virus (pas besoin d'antivirus), pas de bluescreen...
\item Linux marche sur des vieux PC, mais aussi très bien sur les grandes machines
\item Et c'est gratuit, enfin légalement gratuit ;)
\end{itemize}

\section*{Et les logiciels libres?}
Intuitivement, on pense que logiciel libre=freeware.
La vraie définition (par Richard Stallmann) d'un logiciel libre va cependant au delà de la gratuité.
Enfait, un logiciel peut être défini comme libre s'il donne les droits suivants aux utilisateurs :

\begin{itemize}
\item pouvoir l’utiliser comme bon leur semble.
\item pouvoir accéder au code source (la recette) afin de l’étudier et de le comprendre.
\item pouvoir modifier le code source pour n’importe quel usage ou pour l’améliorer.
\item pouvoir redistribuer le logiciel, sous condition que le code source reste accessible.
\end{itemize}

La plupart des étudiants utilisent des logiciels libres, parfois sans le savoir.
Des exemples sont: Firefox, VLC, Gimp, Libre Office (ancien OpenOffice), jDownloader.
Des logiciels qui ne sont pas installés chez vous sur le disque dur mais
que vous accédez seulement au moyen d'un navigateur web sont, eux aussi, souvent libre.
Un exemple que vous conaissez tous est la plateforme iCampus utilisé par l'UCL
ou encore les sites web construits avec Wordpress. Vous voyez donc que les logiciels 
libres sont déployés à large échelle et fonctionnent de manière fiable.

\section*{Et le Kot à projet Louvain-li-Nux alors?}
Notre projet est justement la promotion des logiciels libres et de Linux en particulier sur le site de Louvain-la-Neuve.
Pour ce faire, nous organisons entre autres une install party par quadri (ou on installe Linux sur plein d'ordis),
une foire du libre ou des entreprises qui font leur argent avec des logiciels libres (oui ça existe) se présentent,
une séance d'introduction à \LaTeX{}...

En plus, vous pouvez venir tous les lundis soirs chez nous et on va dépanner/améliorer votre ordinateur.
Pour plus d'informations:
                                                                                                                       
\end{document}
