%%%%%%%%%%%%%%%%%%%%%%%%%%%%%%%%%%%
% Header pour fiche info du LLNux %
%     Version 1.1 (03/11/2012)    %
%%%%%%%%%%%%%%%%%%%%%%%%%%%%%%%%%%%

\documentclass[a4paper,10pt]{article}

\usepackage[utf8]{inputenc}           % UTF-8
\usepackage[a4paper, dvips]{geometry} % mise en page générale
\usepackage{lmodern}
\usepackage[T1]{fontenc}              % polices
\usepackage{multicol}                 % multi-colonnes avancé
\usepackage{xspace}                   % espaces conditionnels
\usepackage{graphicx}                 % images
\usepackage[francais]{babel}          % français

\usepackage{amsmath}                  % math
\usepackage{amssymb}                  % math
\usepackage{gensymb}                  % math

\usepackage{url}
	\urlstyle{sf}
\usepackage[usenames]{color}
\usepackage{varioref}                 % \vpageref

\usepackage{fancyhdr}                 % cfoot
\pagestyle{fancy}                     % we need info at the bottom but not a footnote
\renewcommand{\headrulewidth}{0pt}    % ça crée une ligne au-dessus... on l'enlève
% \thispagestyle{empty}               % we need only one page

\renewcommand{\familydefault}{\sfdefault}

\definecolor{codeBlue}{rgb}{0,0,1}
\definecolor{webred}{rgb}{0.5,0,0}
\definecolor{codeGreen}{rgb}{0,0.5,0}
\definecolor{codeGrey}{rgb}{0.6,0.6,0.6}
\definecolor{webdarkblue}{rgb}{0,0,0.4}
\definecolor{webgreen}{rgb}{0,0.3,0}
\definecolor{webblue}{rgb}{0,0,0.8}
\definecolor{orange}{rgb}{0.7,0.1,0.1}

% Configuration de babel
\NoAutoSpaceBeforeFDP

% Configuration de graphicx
\graphicspath{{images/}}

% Titre, auteur et date
\title{Informations sur les logiciels libres}
\author{Kot-à-projet Louvain-Li-Nux}
\date{\today}

% Commandes personnalisées
\newcommand{\upcirc}{$^{\circ}$\xspace}
\newcommand{\super}[1]{$^{\mbox{\scriptsize #1}}$\xspace}

% Réduction des marges
% Dimensions de la page :       	

  %%%%%%%%%%%%%%%%%%%%%%%%%%%%%%%%%%%%%%%%  0
  %   |                                  %
  %---+----------------------------------%  1
  %   | +----------------------------+   %  2
  %   | |          en-tête           |   %
  %   | +----------------------------+   %  3
  %   | +----------------------------+   %  4
  %   | |                            |   %       Remarques : 
  %   | |                            |   %        . distance de '0' à '1' : un pouce + \voffset
  %   | |                            |   %        . distance de 'a' à 'b' : un pouce + \hoffset
  %   | |           texte            |   %
  %   | |                            |   %
  %   | |                            |   %
  %   | |                            |   %
  %   | +----------------------------+   %  5
  %   | +----------------------------+   %
  %   | |         bas de page        |   %
  %   | +----------------------------+   %  6
  %%%%%%%%%%%%%%%%%%%%%%%%%%%%%%%%%%%%%%%%
  %a  b c                            d

   % général
     \voffset       -40mm    % pour descendre (si positif) ou remonter (si négatif) le tout
     \hoffset       -12.5mm    % pour agrandir (si positif) ou diminuer (si négatif) la marge gauche
     \oddsidemargin 10pt   % distance de 'b' à 'c'
   % texte
     \headsep       42pt   % distance de '3' à '4', la distance entre l'en-tête et le texte
     \textheight    760pt  % distance de '4' à '5', pour déterminer la hauteur du texte
     \textwidth     177.5mm  % distance de 'c' à 'd' 
   % en-tête
     \topmargin     0pt    % distance de '1' à '2', pour descendre (si positif) ou remonter (si négatif) le tout
     \headheight    14pt   % distance de '2' à '3', doit être > 13.59999
   % bas de page
     \footskip      38pt   % distance de '5' à '6', la distance entre le texte et le bas de page
%\addtolength{\hoffset}{-1cm}
%\addtolength{\textwidth}{2.5cm}
%\addtolength{\voffset}{-2cm}
%\addtolength{\textheight}{4.9cm}

% Logo du Louvain-Li-Nux
\newsavebox{\logollnux}
\sbox{\logollnux}{\raisebox{-2.2cm}{\includegraphics[height=2.5cm]{../logo.png}}}
\newsavebox{\logoqr}
\sbox{\logoqr}{\raisebox{-2.2cm}{\includegraphics[height=2.5cm]{../qr.png}}}

%titre
\newcommand{\titlellnux}[1]
{
	\hspace{-1cm}
	\begin{tabular}{lp{13cm}r}
		\usebox{\logoqr}
			&
		\begin{center}{\Large Les fiches du Louvain-li-Nux\linebreak \linebreak
		\LARGE \textbf{Episode #1}}\end{center}
			&
		\usebox{\logollnux}
	\end{tabular}
	\vspace{-.5cm}
}

% pied de page
\rfoot{\url{www.louvainlinux.be}\hspace{-3.0cm}}
\cfoot{}
\lfoot{Rue Constantin Meunier 12 (Bruyères)}

% PDF
   \usepackage[pdftitle={Fiche Informative},  % apparition ds les propriétés du doc
               pdfauthor={Louvain-li-Nux},
               pdfsubject={Fiche Informative du Louvain-li-Nux},
               pdfkeywords={louvain-li-nux,llnux,fiche info,lln,kap,ucl},
	       colorlinks=true,
	       linkcolor=webdarkblue, 
	       filecolor=webblue, 
	       urlcolor=webdarkblue,
	       citecolor=webgreen]{hyperref}


\begin{document}

% Titre
\begin{tabular}{p{13cm}r}
	\begin{center}{\Large Les fiches du Louvain-li-Nux\linebreak \linebreak
	\LARGE Episode 5: Les logiciels libres}\end{center}
		&
	\usebox{\logollnux}
\end{tabular}

\vspace*{0.5cm}
Voici déjà notre 5ème fiche informative. Dans celle-ci vous découvrirez ce que sont les logiciels libres, leurs avantages ainsi qu'une correspondance entre logiciels propriétaire et logiciels libre afin de vous aider à changer.

\section*{Qu'est-ce qu'un logiciel libre?}

Un logiciel libre est caractérisé par plusieurs points : 
\begin{itemize}
\item il est libre d'utilisation: toute personne peut l'utiliser à toutes les fins possible
\item il est libre d'étude: son code source (recette de fabrication) est accessible par tout ceux qui veulent
\item il est libre de modification: toute personne désirant ajouter un ingrédient dans la recette peut le faire
\item il est libre de redistribution: toute personne peut recréer le même programme et le distribuer
\end{itemize}

Pour éviter qu'un logiciel libre passe du coté obscur de la force, il est également soumis à une licence d'utilisation. Il en existe plusieurs dont la licence CC ("Creative Commons") ou la licence GPL ("General Public Licence"), pour la plupart regroupées sous le principe Copyleft. Chacune de ces licences autorisent un certain type d'utilisation. Pour toutes oeuvres regroupées sous Copyleft, tout dérivé est obligé de rester sous licence Copyleft, cela permet ainsi de préserver le coté libre d'une oeuvre.

\section*{Quelques avantages d'un logiciel libre}

Les logiciels libres ont de nombreux avantage.

Comme un logiciel libre est le résultat du travail de nombreuse personnes, il en est d'autant plus efficace. Puisque chacun à la possibilité de corriger les bugs, ceux-ci sont d'autant plus vite éliminé. Si il manque, une fonctionnalité, celle-ci sera rapidement rajoutée. Et comme le nombre de développeur est élevé et qu'ils ont tous des besoins spécifiques, cela assure au programme un très grand nombre de fonctionnalité avancées.

De plus, on est assuré d'une sécurité au point de vue des données. Toutes failles de sécurité sont également corrigée très rapidement.

Ensuite, les logiciels libres respectent les standards. Ainsi, aucun logiciel libre n'ira faire des sauvegarde dans un format illisible par d'autre logiciel. Cela permet de conserver la lisibilité dans les donnée même si le logiciel d'origine n'existe plus.

Troisièmement, la plupart des logiciels libres sont gratuits.

\section*{Quelques logiciels propriétaires et leurs équivalents libres...}

Je vais maintenant vous donner une petite liste de logiciels propriétaires avec leur équivalent libres.
\begin{itemize}
\item La suite M\$ office remplaçable par la suite LibreOffice
\item Outlook Express remplaçable par Mozilla Thunderbird
\item On peut utiliser Gimp à la place du célèbre logiciel Photoshop
\item Comme navigateur web, à la place de Google Chrome, Safari ou encore Internet Explorer, vous pouvez utiliser Chomium
\item Comme lecteur vidéo, à la place de Windows Media Player et RealPlayer, il y a VLC
\item A la place d'Adobe Reader, vous pouvez utiliser Xpdf pour lire vos PDF
\item Pour ceux qui font de la modélisation 3D sur Maya, Blender est la pour vous
\item Avis aux possesseurs d'iPod, au lieu de passer par iTunes, vous pouvez utiliser Clementine
\item Pour les utilisateurs de Winrar, il existe 7-zip (Windows), ou p7zip (Linux)
\item Matlab à également un équivalent, Octave

\end{itemize}


\end{document}
	
