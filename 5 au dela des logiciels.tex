%%%%%%%%%%%%%%%%%%%%%%%%%%%%%%%%%%%
% Header pour fiche info du LLNux %
%     Version 1.1 (03/11/2012)    %
%%%%%%%%%%%%%%%%%%%%%%%%%%%%%%%%%%%

\documentclass[a4paper,10pt]{article}

\usepackage[utf8]{inputenc}           % UTF-8
\usepackage[a4paper, dvips]{geometry} % mise en page générale
\usepackage{lmodern}
\usepackage[T1]{fontenc}              % polices
\usepackage{multicol}                 % multi-colonnes avancé
\usepackage{xspace}                   % espaces conditionnels
\usepackage{graphicx}                 % images
\usepackage[francais]{babel}          % français

\usepackage{amsmath}                  % math
\usepackage{amssymb}                  % math
\usepackage{gensymb}                  % math

\usepackage{url}
	\urlstyle{sf}
\usepackage[usenames]{color}
\usepackage{varioref}                 % \vpageref

\usepackage{fancyhdr}                 % cfoot
\pagestyle{fancy}                     % we need info at the bottom but not a footnote
\renewcommand{\headrulewidth}{0pt}    % ça crée une ligne au-dessus... on l'enlève
% \thispagestyle{empty}               % we need only one page

\renewcommand{\familydefault}{\sfdefault}

\definecolor{codeBlue}{rgb}{0,0,1}
\definecolor{webred}{rgb}{0.5,0,0}
\definecolor{codeGreen}{rgb}{0,0.5,0}
\definecolor{codeGrey}{rgb}{0.6,0.6,0.6}
\definecolor{webdarkblue}{rgb}{0,0,0.4}
\definecolor{webgreen}{rgb}{0,0.3,0}
\definecolor{webblue}{rgb}{0,0,0.8}
\definecolor{orange}{rgb}{0.7,0.1,0.1}

% Configuration de babel
\NoAutoSpaceBeforeFDP

% Configuration de graphicx
\graphicspath{{images/}}

% Titre, auteur et date
\title{Informations sur les logiciels libres}
\author{Kot-à-projet Louvain-Li-Nux}
\date{\today}

% Commandes personnalisées
\newcommand{\upcirc}{$^{\circ}$\xspace}
\newcommand{\super}[1]{$^{\mbox{\scriptsize #1}}$\xspace}

% Réduction des marges
% Dimensions de la page :       	

  %%%%%%%%%%%%%%%%%%%%%%%%%%%%%%%%%%%%%%%%  0
  %   |                                  %
  %---+----------------------------------%  1
  %   | +----------------------------+   %  2
  %   | |          en-tête           |   %
  %   | +----------------------------+   %  3
  %   | +----------------------------+   %  4
  %   | |                            |   %       Remarques : 
  %   | |                            |   %        . distance de '0' à '1' : un pouce + \voffset
  %   | |                            |   %        . distance de 'a' à 'b' : un pouce + \hoffset
  %   | |           texte            |   %
  %   | |                            |   %
  %   | |                            |   %
  %   | |                            |   %
  %   | +----------------------------+   %  5
  %   | +----------------------------+   %
  %   | |         bas de page        |   %
  %   | +----------------------------+   %  6
  %%%%%%%%%%%%%%%%%%%%%%%%%%%%%%%%%%%%%%%%
  %a  b c                            d

   % général
     \voffset       -40mm    % pour descendre (si positif) ou remonter (si négatif) le tout
     \hoffset       -12.5mm    % pour agrandir (si positif) ou diminuer (si négatif) la marge gauche
     \oddsidemargin 10pt   % distance de 'b' à 'c'
   % texte
     \headsep       42pt   % distance de '3' à '4', la distance entre l'en-tête et le texte
     \textheight    760pt  % distance de '4' à '5', pour déterminer la hauteur du texte
     \textwidth     177.5mm  % distance de 'c' à 'd' 
   % en-tête
     \topmargin     0pt    % distance de '1' à '2', pour descendre (si positif) ou remonter (si négatif) le tout
     \headheight    14pt   % distance de '2' à '3', doit être > 13.59999
   % bas de page
     \footskip      38pt   % distance de '5' à '6', la distance entre le texte et le bas de page
%\addtolength{\hoffset}{-1cm}
%\addtolength{\textwidth}{2.5cm}
%\addtolength{\voffset}{-2cm}
%\addtolength{\textheight}{4.9cm}

% Logo du Louvain-Li-Nux
\newsavebox{\logollnux}
\sbox{\logollnux}{\raisebox{-2.2cm}{\includegraphics[height=2.5cm]{../logo.png}}}
\newsavebox{\logoqr}
\sbox{\logoqr}{\raisebox{-2.2cm}{\includegraphics[height=2.5cm]{../qr.png}}}

%titre
\newcommand{\titlellnux}[1]
{
	\hspace{-1cm}
	\begin{tabular}{lp{13cm}r}
		\usebox{\logoqr}
			&
		\begin{center}{\Large Les fiches du Louvain-li-Nux\linebreak \linebreak
		\LARGE \textbf{Episode #1}}\end{center}
			&
		\usebox{\logollnux}
	\end{tabular}
	\vspace{-.5cm}
}

% pied de page
\rfoot{\url{www.louvainlinux.be}\hspace{-3.0cm}}
\cfoot{}
\lfoot{Rue Constantin Meunier 12 (Bruyères)}

% PDF
   \usepackage[pdftitle={Fiche Informative},  % apparition ds les propriétés du doc
               pdfauthor={Louvain-li-Nux},
               pdfsubject={Fiche Informative du Louvain-li-Nux},
               pdfkeywords={louvain-li-nux,llnux,fiche info,lln,kap,ucl},
	       colorlinks=true,
	       linkcolor=webdarkblue, 
	       filecolor=webblue, 
	       urlcolor=webdarkblue,
	       citecolor=webgreen]{hyperref}


\begin{document}

% Titre
\begin{tabular}{p{14cm}r}
    \begin{center}{\Large Les fiches du Louvain-li-Nux\linebreak \linebreak
    \LARGE \textbf{Episode 5: Le libre, pas que des logiciels}}\end{center}
		&
	\usebox{\logollnux}
\end{tabular}

% Contenu
Le monde du libre ne se limite aucunement aux logiciels ou même à l'informatique en général.\\
Parcourons certains exemples plus ou moins connus pour illustrer ce que nous entendons par cela.

\paragraph{Wikipedia}
Tout le monde l'utilise de temps en temps. Cette encyclopédie libre a été fondée par Jimmy Wales en 2001.
Chacun peut contribuer aux articles; d'ou la taille phénoménale.
On y trouve plus d'un million d'articles en français et plus de quatre en anglais, soit plus que toutes les encyclopédies commerciales combinées.
La critique principale est que le contenu serait non fiable.
Est-de vraiment le cas? C'est vrai qu'on ne connait pas toujours les auteurs des articles, mais bien leurs sources respectives.
C'est là dessus que se base le contrôle du contenu et franchement, ça marche.
S'il convient de critiquer la Wikipedia, c'est probablement parce que les articles ont tendance à devenir trop longs et que des sujets tel que l'informatique ou les séries TV
sont largement surreprésentées en raison des intérêts personnels des auteurs.

En plus des encyclopédies, il existe aussi \textit{Wikimedia Commons} une base de données multimédia libre
qui est la source des illustrations que vous trouvez dans les articles.
Parmi les autres projets liés, on trouve \textit{Wikiquote}, un recueil de citations et \textit{Wiktionnaire}, un dictionnaire universel.

\paragraph{Open Street Map}
Le principe est le même que pour Wikipedia, tout le monde peut aider à construire ces cartes.
Faites le test et comparez les avec ceux de Google Maps ou encore des cartes imprimées.
Dans la plupart des cas, les cartes de \textit{Open Street Map} seront beaucoup plus détaillées.
Il reste encore à développer un algorithme d'optimisation de chemin.

\paragraph{La musique libre}
Il y en a de deux genres. Parlons d'abord des \textit{Creative Commons (CC)}. Ce sont des licences grâce auxquelles l'auteur de la pièce (musique ou autre) peut vous permettre
d'utiliser son \oe uvre tant que certaines conditions restent remplies. Par exemple, la mention \textit{BY} veut dire qu'il faut mentionner l'auteur original, \textit{NC} permet l'utilisation uniquement dans des circonstances non commerciales et \textit{SA} vous permet de réutiliser ce matériau tant que votre nouveau produit reste sous la même licence libre. On parle parfois d'une licence \textit{copyleft} puisqu'elle utilise les mêmes mécanismes juridiques que le \textit{copyright} pour atteindre le but contraire. Vous trouvez plein de musique libre sur \url{www.jamendo.com}

En plus de cela, il y a la musique dont le copyright a expiré. Actuellement, c'est le cas \textit{70 ans après la mort de l'auteur}.
Il n'est pas très utile de retenir ce chiffre puisqu'il se fait scandaleusement prolonger à chaque fois que des \oe uvres comme Micky Mouse ou les chansons d'Elvis risquent de tomber dans le \textit{public domain}.
Néanmoins, une chose est sure. La plupart du patrimoine culurel qu'est la musique classique est aujourd'hui libre de droits d'auteur.
C'est entre autres pour cette raison qu'on en entend autant dans la publicité et les films.
C'est uniquement la pièce de musique en soi mais pas l'interprétation par un orchestre particulier qui est libre.
Le site web \url{www.musopen.org} présente donc un catalogue de musique classique rejoué avec l'intention de rendre ces interprétations libres.
Nous vous conseillons de les télécharger dans le format \textit{flac} (free lossless audio codec) en raison de la qualité supérieure aux mp3.

\paragraph{Les livres}
Ici on trouve de nouveau les scans de livres historiques sous public domain mais aussi quelques romans publiés sous licence \textit{CC}.
Mentionnons à titre d'exemple les romans policiers et de science fiction de \textit{Cory Doctorow}.

\paragraph{Les cours}
Ceci est particulièrement intéressant pour tout étudiant: Presque tous les universités de rénommé mondial mettent une grande partie de leurs cours et conférence en ligne.
Ces vidéos sont typiquement sous licence \textit{CC}. Harvard, Oxford, MIT, Yale, Princeton, HEC Paris, Stanford, Berkeley... Tous les grands noms y sont.
La distribution se fait soit sur le site web de l'université, soit au moyen de flux RSS, soit dans itunesU, il y en a aussi pas mal sur Youtube etc.
Notre Université Catholique de Louvain est un peu en retard, mais ça commence aussi chez nous.

\paragraph{Internet archive}
Sur le site \url{www.archive.org} vous trouvez une collection de presque tous ces \oe uvres libres mentionnées ci dessus.
La base de données contient entre autres 500 000 films, un million d'enregistremetns audio et trois millions de livres.
Si vous voulez par exemple revoir comment les chaines TV ont réagi lors des évènements du 11.09.2001, vous le trouvez ici.
Pour alléger le trafic sur leurs serveurs, l'archive fait usage du protocol \textit{Torrent} que vous conaissez tous.

\textit{Internet archive} s'est aussi donnée le but d'archiver le web. Le \textit{Wayback machine} permet ainsi de consulter des sites web tels qu'ils étaient à des moments précis dans le passé.

\paragraph{Sintel}
est un court métrage libre de 15 minutes réalisé en 2010 par Tom Rosendaal et toute une équipe d'artistes.
L'histoire tourne autour d'un dragon orphelin \textit{Scales} et de \textit{Sintel} qui s'en occupe jusqu'à ce qu'il se fait enlever par un dragon adulte.
Nous ne racontons pas la fin. Le logiciel libre \textit{Blender} est à la source des images animées.
Vous pouvez télécharger le film gratuitement (et légalement) en FullHD, même en 4K pour les spécialistes.

\end{document}
	
		
	
