%%%%%%%%%%%%%%%%%%%%%%%%%%%%%%%%%%%
% Header pour fiche info du LLNux %
%     Version 1.1 (03/11/2012)    %
%%%%%%%%%%%%%%%%%%%%%%%%%%%%%%%%%%%

\documentclass[a4paper,10pt]{article}

\usepackage[utf8]{inputenc}           % UTF-8
\usepackage[a4paper, dvips]{geometry} % mise en page générale
\usepackage{lmodern}
\usepackage[T1]{fontenc}              % polices
\usepackage{multicol}                 % multi-colonnes avancé
\usepackage{xspace}                   % espaces conditionnels
\usepackage{graphicx}                 % images
\usepackage[francais]{babel}          % français

\usepackage{amsmath}                  % math
\usepackage{amssymb}                  % math
\usepackage{gensymb}                  % math

\usepackage{url}
	\urlstyle{sf}
\usepackage[usenames]{color}
\usepackage{varioref}                 % \vpageref

\usepackage{fancyhdr}                 % cfoot
\pagestyle{fancy}                     % we need info at the bottom but not a footnote
\renewcommand{\headrulewidth}{0pt}    % ça crée une ligne au-dessus... on l'enlève
% \thispagestyle{empty}               % we need only one page

\renewcommand{\familydefault}{\sfdefault}

\definecolor{codeBlue}{rgb}{0,0,1}
\definecolor{webred}{rgb}{0.5,0,0}
\definecolor{codeGreen}{rgb}{0,0.5,0}
\definecolor{codeGrey}{rgb}{0.6,0.6,0.6}
\definecolor{webdarkblue}{rgb}{0,0,0.4}
\definecolor{webgreen}{rgb}{0,0.3,0}
\definecolor{webblue}{rgb}{0,0,0.8}
\definecolor{orange}{rgb}{0.7,0.1,0.1}

% Configuration de babel
\NoAutoSpaceBeforeFDP

% Configuration de graphicx
\graphicspath{{images/}}

% Titre, auteur et date
\title{Informations sur les logiciels libres}
\author{Kot-à-projet Louvain-Li-Nux}
\date{\today}

% Commandes personnalisées
\newcommand{\upcirc}{$^{\circ}$\xspace}
\newcommand{\super}[1]{$^{\mbox{\scriptsize #1}}$\xspace}

% Réduction des marges
% Dimensions de la page :       	

  %%%%%%%%%%%%%%%%%%%%%%%%%%%%%%%%%%%%%%%%  0
  %   |                                  %
  %---+----------------------------------%  1
  %   | +----------------------------+   %  2
  %   | |          en-tête           |   %
  %   | +----------------------------+   %  3
  %   | +----------------------------+   %  4
  %   | |                            |   %       Remarques : 
  %   | |                            |   %        . distance de '0' à '1' : un pouce + \voffset
  %   | |                            |   %        . distance de 'a' à 'b' : un pouce + \hoffset
  %   | |           texte            |   %
  %   | |                            |   %
  %   | |                            |   %
  %   | |                            |   %
  %   | +----------------------------+   %  5
  %   | +----------------------------+   %
  %   | |         bas de page        |   %
  %   | +----------------------------+   %  6
  %%%%%%%%%%%%%%%%%%%%%%%%%%%%%%%%%%%%%%%%
  %a  b c                            d

   % général
     \voffset       -40mm    % pour descendre (si positif) ou remonter (si négatif) le tout
     \hoffset       -12.5mm    % pour agrandir (si positif) ou diminuer (si négatif) la marge gauche
     \oddsidemargin 10pt   % distance de 'b' à 'c'
   % texte
     \headsep       42pt   % distance de '3' à '4', la distance entre l'en-tête et le texte
     \textheight    760pt  % distance de '4' à '5', pour déterminer la hauteur du texte
     \textwidth     177.5mm  % distance de 'c' à 'd' 
   % en-tête
     \topmargin     0pt    % distance de '1' à '2', pour descendre (si positif) ou remonter (si négatif) le tout
     \headheight    14pt   % distance de '2' à '3', doit être > 13.59999
   % bas de page
     \footskip      38pt   % distance de '5' à '6', la distance entre le texte et le bas de page
%\addtolength{\hoffset}{-1cm}
%\addtolength{\textwidth}{2.5cm}
%\addtolength{\voffset}{-2cm}
%\addtolength{\textheight}{4.9cm}

% Logo du Louvain-Li-Nux
\newsavebox{\logollnux}
\sbox{\logollnux}{\raisebox{-2.2cm}{\includegraphics[height=2.5cm]{../logo.png}}}
\newsavebox{\logoqr}
\sbox{\logoqr}{\raisebox{-2.2cm}{\includegraphics[height=2.5cm]{../qr.png}}}

%titre
\newcommand{\titlellnux}[1]
{
	\hspace{-1cm}
	\begin{tabular}{lp{13cm}r}
		\usebox{\logoqr}
			&
		\begin{center}{\Large Les fiches du Louvain-li-Nux\linebreak \linebreak
		\LARGE \textbf{Episode #1}}\end{center}
			&
		\usebox{\logollnux}
	\end{tabular}
	\vspace{-.5cm}
}

% pied de page
\rfoot{\url{www.louvainlinux.be}\hspace{-3.0cm}}
\cfoot{}
\lfoot{Rue Constantin Meunier 12 (Bruyères)}

% PDF
   \usepackage[pdftitle={Fiche Informative},  % apparition ds les propriétés du doc
               pdfauthor={Louvain-li-Nux},
               pdfsubject={Fiche Informative du Louvain-li-Nux},
               pdfkeywords={louvain-li-nux,llnux,fiche info,lln,kap,ucl},
	       colorlinks=true,
	       linkcolor=webdarkblue, 
	       filecolor=webblue, 
	       urlcolor=webdarkblue,
	       citecolor=webgreen]{hyperref}


\begin{document}

% Titre
\begin{tabular}{p{13cm}r}
	\begin{center}{\Large Les fiches du Louvain-li-Nux\linebreak \linebreak
	\LARGE Episode 3: Le travail collaboratif}\end{center}
		&
	\usebox{\logollnux}
\end{tabular}

% Contenu

\section*{Les ennuis du travail en groupe}
Bien que nécessaire dans l'environnement de travail au XXI ième siècle, le travail
en groupe n'est pas toujours évident à mettre en \oe uvre. 
Vous avez tous déjà rendu de tels travaux et vous connaissez donc les problèmes récurrents.\\

Souvent il y a un membre sur-motivé qui insiste à faire tout tout seul, ou encore
un ou plusieurs qui se reposent sur le dos des autres. Même si tout le monde est environ
sur un pied d'égalité, les problèmes persistent. \\

Si vous travaillez tous avec Word dans vos coins, vous vous retrouvez très vite avec une pléthore de fichiers différents, avec des formats différents (\textit{doc}, \textit{docx}, \textit{odt}, \textit{txt}, \textit{xls}, \textit{pdf}, mails, etc.).
Bien vite, vous ne saurez plus quelles versions prendre pour la mise en commun.\\
D'un autre côté, séparé le travail en blocs complètement distincts pourrait aider mais le risque pourrait être que personne ne se sente responsable pour le texte en entier.
Au pire, vous aurez un mélange de fichiers avec des formats différents demandant l'emploi de plusieurs logiciels, etc. Bonne chance!

Si vous travaillez avec un outil comme \textit{Dropbox}, il est fréquent de voir des fichiers effacés, modifiés ou revenu dans une version antérieur sans s'en rendre de compte puisque la dernière modification écrasera le travail des autres. Que faire alors?

\section*{Un exemple avec des projets de logiciels libres ?}
En fait, les programmeurs étaient confrontés aux mêmes soucis mais 
à une échelle plus grande  en informatique. Allons voir comment ils se débrouillent.

Tous derrière son pc... parfois très éloignés et pas souvent en contact. Ex: avec Linux et les milliers de contributeurs.

Une différence aux rapports: En informatique c'est souvent une bonne idée de diviser le travail en blocs qui communiquent uniquement au moyen d'interfaces préfinis

\section*{Comment améliorer la productivité d'un travail de groupe}

\subsection*{Création d'un rapport}

\subsection*{La programmation à plusieurs}

\subsection*{Quelques outils utiles}
                                                                                                                   
\end{document}
	
